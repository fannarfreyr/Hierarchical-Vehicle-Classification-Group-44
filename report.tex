\documentclass[11pt, a4paper]{article}

% --- UNIVERSAL PREAMBLE BLOCK ---
% Set standard academic margins
\usepackage[a4paper, top=2.5cm, bottom=2.5cm, left=2cm, right=2cm]{geometry}
% Required for modern font handling
\usepackage{fontspec}

% Setup language (English is the default)
\usepackage[english, bidi=basic, provide=*]{babel}
\babelprovide[import, onchar=ids fonts]{english}

% Set default font to Serif for an academic tone (using Noto Serif which is available in the compiler)
\babelfont{rm}{Noto Serif}

% Required for advanced math environments (equations, matrices)
\usepackage{amsmath}
% Required for high-quality table formatting (toprule, midrule, bottomrule)
\usepackage{booktabs}
% Required for links in the final PDF
\usepackage{hyperref}
% --- END UNIVERSAL PREAMBLE BLOCK ---

\title{Deep Learning Project Report: Semantic Segmentation with CNNs}
\author{Your Name(s) / Team Name}
\date{\today}

\begin{document}

\maketitle

\begin{abstract}
This report details the implementation and evaluation of a Convolutional Neural Network (CNN) designed for the task of semantic image segmentation. The network, based on a modified U-Net architecture, was trained on a relevant dataset. Our experimental results show a competitive performance, demonstrating the model's efficacy for dense prediction tasks.
\end{abstract}

\section{Introduction}
Deep learning has shown remarkable success in computer vision. Specifically, the task of \textbf{semantic segmentation}, which assigns a class label to every pixel, is crucial for applications like autonomous navigation and medical diagnostics. This project focuses on designing and optimizing a segmentation model.

\section{Methodology}
\subsection{Model Architecture}
We utilize the U-Net architecture, known for its encoder-decoder structure and skip connections. The encoder gradually reduces spatial dimensions while increasing feature depth, and the decoder reconstructs the segmentation mask.

\subsection{Loss Function}
The training process uses the combined Dice-Jaccard loss. The Dice coefficient ($D$) is commonly used for imbalanced segmentation tasks:
$$
D = \frac{2 |P \cap G|}{|P| + |G|}
$$
The final loss function is derived from the Jaccard Index (IoU), which is related to the Dice coefficient, and is optimized using the Adam optimizer.

\section{Results and Discussion}
The model was trained for 50 epochs. Performance was measured using the Mean Intersection over Union (mIoU).

\begin{table}[htbp]
    \centering
    \caption{Key Segmentation Performance Metrics}
    \label{tab:results}
    \begin{tabular}{lcc}
        \toprule
        Metric & Value & Baseline Benchmark \\
        \midrule
        mIoU (\%) & $78.5$ & $80.2$ \\
        Pixel Accuracy (\%) & $95.1$ & $96.0$ \\
        \bottomrule
    \end{tabular}
\end{table}

Our achieved mIoU of $78.5\%$ is robust, considering the complexity of the dataset.

\section{Conclusion}
The deep learning model successfully addresses the semantic segmentation task. Future work will focus on integrating self-attention mechanisms to improve boundary precision.

\end{document}